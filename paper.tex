\documentclass[11pt]{article}
\usepackage[round,comma,authoryear]{natbib}
\usepackage{fullpage}
\usepackage{authblk}
\usepackage{graphicx}
\usepackage{booktabs}

\usepackage[newfloat]{minted}
\usepackage{tcolorbox}
\usepackage{etoolbox}
\BeforeBeginEnvironment{minted}{\begin{tcolorbox}}
\AfterEndEnvironment{minted}{\end{tcolorbox}}
\usepackage{url}
\usepackage{fancyvrb}
\usepackage{caption}
\newenvironment{code}{\captionsetup{type=listing}\centering}{}
\SetupFloatingEnvironment{listing}{name=Demographic Model}
% can either input code directly in a minted environment
% or input from a source file using \inputminted

% local definitions
\newcommand{\ms}[0]{\texttt{ms}}
\newcommand{\msprime}[0]{\texttt{msprime}}
\newcommand{\stdpopsim}[0]{\texttt{stdpopsim}}
\newcommand{\demes}[0]{\texttt{demes}}
\newcommand{\Demes}[0]{\texttt{Demes}}
\newcommand{\moments}[0]{\texttt{moments}}
\newcommand{\dadi}[0]{\texttt{$\partial$a$\partial$i}}
\newcommand{\fwdpy}[0]{\texttt{fwdpy11}}
\newcommand{\slim}[0]{\texttt{SLiM}}
\newcommand{\gadma}[0]{\texttt{GADMA}}

\newcommand{\aprcomment}[1]{{\textcolor{blue}{APR: #1}}}
\newcommand{\jkcomment}[1]{{\textcolor{red}{JK: #1}}}

\begin{document}

\title{Demes: a specification for interchanging complex demography}
% First authors
\author[1,$\star$]{Graham Gower}
\author[2,$\star$]{Aaron P. Ragsdale}

% Authors: please add your name and affiliation here, in alphabetical order.
% Second Authors
\author[3]{Ryan N. Gutenkunst}
\author[4]{Matthew Hartfield}
\author[5]{Ekaterina Noskova}
\author[3]{Travis J. Struck}
%Senior Authors
\author[6,$\dagger$,$\aleph$]{Jerome Kelleher}
\author[7,$\dagger$]{Kevin R. Thornton}

\affil[1]{Lundbeck GeoGenetics Centre, Globe Institute, University of Copenhagen}
\affil[2]{FIXME: Aaron's affiliation}
\affil[3]{Department of Molecular and Cellular Biology, University of Arizona}
\affil[4]{Institute of Evolutionary Biology, The University of Edinburgh}
\affil[5]{Computer Technologies Laboratory, ITMO University}
\affil[6]{Big Data Institute, Li Ka Shing Centre for Health Information and
    Discovery, University of Oxford}
\affil[7]{Ecology and Evolutionary Biology, University of California at Irvine}

\affil[$\star$]{Denotes shared first authorship, listed alphabetically}
\affil[$\dagger$]{Denotes shared senior authorship, listed alphabetically}
\affil[$\aleph$]{Denotes corresponding author}


\maketitle

\abstract{
Understanding the demographic history of populations is a
key goal in population genetics, and with improving methods
and data, ever more complex models are being estimated.
Demographic models of current interest (consisting of a set of discrete populations,
their sizes and growth rates, and continuous and pulse migrations
between those populations over a number of epochs) can require
dozens of parameters to fully describe. There is currently
no standard format to define such models, significantly
hampering  progress in the field. In particular, the important
task of translating the model descriptions in published
work into input suitable for population genetic simulators is labour intensive
and error prone.
We propose the Demes data model and file format,
built on widely used technologies,
to alleviate these issues. Demes provides a well-defined and unambigous
graphical model of populations and their properties that is straightforward to
implement in software, and a text file format that is designed for
simplicity and clarity.
We provide thoroughly tested implementations of Demes parsers in Python
and C, and showcase initial support in a number of simulators and inference
methods.
}

\section*{Introduction}

The ever-increasing amount of genetic sequencing data from genetically and
geographically diverse species and populations has allowed us to infer complex
demography and study life history at fine scales. An integral component to such
population genetics studies is simulation. Software to either simulate whole
genome sequences
\citep{thornton2014cpp,staab2015scrm,kelleher2016efficient,haller2019slim}
or informative summary statistics of diversity
\citep{gutenkunst2009inferring,kamm2017efficient,jouganous2017inferring} have
enabled the increasing complexity of genomic studies, with several software
packages capable of working with large sample sizes, many interacting populations, and
deviations from random-mating models of panmixia.
This ability to infer and simulate such complex demographic scenarios, however,
has highlighted a major shortcoming in community standards. The fragmented
landscape of different approaches to describing demographic models makes
it difficult to compare inferences made by different methods,
and to reliably simulate from previously inferred models.

Inference methods for demography usually output a set of
text files which describe the inferred model,
and nearly all methods use different file formats.
% Thus, any comparison
% between inferences requires these file formats to be parsed, with the user
% being left with the task of unifying the data models.
Inference results are typically reported in publications
via a combination of visual depiction,
a list of key parameters in tabular form and a discussion within the text.
Unfortunately there is often ambiguity in these descriptions and
implementing the precise model inferred for later simulation
is at best tedious and error prone~\citep{ragsdale2020lessons}
and occasionally impossible because of missing information.
The \stdpopsim\ project is a community-maintained catalog of species
information and demographic models designed to provide standardised
population genetic simulations~\citep{adrion2020community}.
The development process for adding demographic models to
\stdpopsim\ tries to avoid errors in model implementation by requiring two
independent implementations of each model
to be equivalent. While this process is
robust and has uncovered both significant
errors~\citep{ragsdale2020lessons} and ambiguities in
published models, it is labour intensive.
It would be far better if published results
were reported in an unambiguous standardised format that could be provided
directly as input to simulators.

Simulation is a core tool in population genetics, and a
large number of different methods have been developed
over the past three decades~\citep{carvajal2008simulation,liu2008survey,
arenas2012simulation,yuan2012overview,hoban2012computer}.
Simulations are based on highly idealised population models
and one of the key uses of inferred demographic histories
is to make simulations more realistic.
Simulation methods take three broad approches to specifying
the demographic model to simulate,
using either a command line
interface~\citep[e.g.][]{hudson2002generating,kern2016discoal},
a custom input file
format~\citep[e.g.][]{guillaume2006nemo,excoffier2011fastsimcoal,shlyakhter2014cosi2},
or an Application Programming Interface (API) to allow
models be defined programmatically~\citep[e.g.][]{
thornton2014cpp,kelleher2016efficient,becheler2019quetzal,haller2019slim}.
Command line interfaces are a concise way of expressing
demographic models, and the syntax defined by \ms~\citep{hudson2002generating}
is used by a number of different
simulators~\citep[e.g.][]{ewing2010msms,chen2009fast,staab2015scrm},
and the POPDemog visualisation method~\citep{zhou2018popdemog}.
However, this conciseness means that models of even intermediate complexity
are difficult for humans to understand, making errors highly likely.
Input parameter file formats for simulators allow the model specification
to be less terse and (potentially) also allow for documentation in the
form of comments. However, there is no standard format shared across
simulators, and the details of the demographic models are not separated
from other simulation parameters. The APIs use by individual simulators
to define demographic models are also not appropriate as an interchange
standard because of the dependency on both the programming language involved,
and the specific details of the simulator in question.

Here we present ``Demes'', a data model and file format
specification for complex demographic models developed by the PopSim
Consortium. The Demes data model precisely defines key parameters
of populations and their relationships over time in an extensible way,
and provides a way to explicitly encode all information relevant to
demography while avoiding repetition. This data model is implemented
in the widely used YAML format~\citep{ben2009yaml}, which provides a
good balance between human and machine readability.
The specification precisely
defines the required behaviour of implementations, ensuring that there
is no ambiguity of interpretation, and includes both a reference
implementation and an extensive suite of test examples and their expected
output. The initial software ecosystem includes high-quality
Python and C parser implementations, as well as utilities for verification
and visualisation of Demes models, and has been implemented in
several popular inference and simulation methods
(Table~\ref{tab-software}).
We hope that this
data model and file format will be widely adopted by the community,
such that users can expect to simulate directly from
inferred models with little or no programming effort.

\section*{Demes}

\begin{itemize}
    \item Demographic models are comprised of one or more interacting
        populations, along with description, doi, time units and generation
        times.
    \item Populations (or demes) are treated as discrete collection of
        exchangeable individuals following a well-defined set of rules
    \item These rules include population sizes, times of existence, and other
        information about the mating system (e.g. selfing or cloning rates)
    \item Relationships between demes are defined by ancestors/descendents
        and possible continuous or instantaneous migration between coexisting
        demes
\end{itemize}

\subsection*{High-level model specification in YAML}

\begin{itemize}
    \item YAML allows us to write this information in a human-readable and
        structured manner (lists, keys/items, etc)
    \item Implicit values, so that models can be written compactly
    \item Example of an IM Model~\ref{code:im_model}
\end{itemize}

\begin{code}
\begin{tcolorbox}
\inputminted[linenos,numbersep=5pt]{yaml}{models/IM.yaml}
\end{tcolorbox}
\captionof{listing}{An IM model.}
\label{code:im_model}
\end{code}

\subsection*{Unambiguous low-level representation}

\begin{itemize}
    \item The Python API, which can read in a YAML file or be built using the
        Builder
    \item Demographic model validitation
    \item Unambiguous, redundant information
    \item Designed to be programatically read by simulation software backends
\end{itemize}

% Add a bit more space between rows.
\renewcommand{\arraystretch}{1.5}
\begin{table}
\begin{center}
\begin{tabular}{lp{12cm}}
\toprule
\multicolumn{2}{l}{Software infrastructure}\\
\midrule
\texttt{demes-python} &
    A Python library for loading, saving, and working with
    Demes models. Includes conversion utilities for existing
    demographic model descriptions such as
    \texttt{ms}~\citep{hudson2002generating}.\\

\texttt{demes-c} &
    A C library for parsing Demes YAML descriptions.
    (\url{https://github.com/grahamgower/demes-c.git}). \\

\texttt{demesdraw} &
    A Python library for visualising Demes models
    (\url{https://github.com/grahamgower/demesdraw.git}). \\

\midrule
\multicolumn{2}{l}{Methods using Demes as input/output format}\\
\midrule

% TOOL AUTHORS: please add a row to this table to summarise what the software
% does, and how it implements demes.

\dadi &
    Summary of what dadi does and how it implements Demes~\citep{gutenkunst2009inferring}.
    \\

\fwdpy &
    Forward-time Wright-Fisher simulation \citep{thornton2014cpp}\\

\gadma &
    Demographic inference \citep{noskova2020gadma}.\\

\moments &
    Compute expected diversity statistics (SFS, LD) and perform
    demographic inference using summary statistics
    \citep{jouganous2017inferring,ragsdale2019models}.\\

\msprime &
    Coalescent simulation of Demes models, including command line
    interface~\citep{kelleher2016efficient,kelleher2020coalescent}.\\

\stdpopsim &
    A community-maintained collection of empirical data for species
    and demographic models specified in Demes
    format~\citep{adrion2020community}.\\

\bottomrule
\end{tabular}
\end{center}
\caption{\label{tab-software}
Software support for Demes. We have included software infrastructure developed
for working with Demes models (parsing, validation, visualisation, etc)
as well as all known downstream software that implements the specification,
at the time of writing.}
\end{table}

\section*{Discussion}

This is an unstructured list of things we should mention:

\begin{itemize}
    \item Recent developments in software engineering have favoured
    static, declarative configuration formats. It feels like a good idea to
    have a Turing complete configuration format, but it turns out not to be.

    \item Drawing parameters from distributions. Why we are static.
    % https://github.com/popsim-consortium/demes-paper/issues/3

    \item Why we've avoided turning this into a model of everything, allowing
    for the specification of genome properties like recombination rates,
    etc.

    \item
    % JK: I thought there was another simulation frontend somewhere..
    Previous efforts such as \texttt{coala}~\citep{staab2016coala} have focused on
    abstracting the differences between individual simulators by providing
    a frontend.

    \item
    % https://github.com/popsim-consortium/demes-paper/issues/11
    We should mention coasim~\citep{mailund2005coasim} because its model
    seems quite close to what we have. It's not suitable as a standard though
    because it uses a programming language.

\end{itemize}

\bibliographystyle{plainnat}
\bibliography{paper}

\end{document}

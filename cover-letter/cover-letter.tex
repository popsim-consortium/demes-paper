\documentclass[11pt]{article}
\usepackage[round]{natbib}

\usepackage{fullpage}
\usepackage{url}

\usepackage{parskip}
\setlength{\parindent}{0em}
\setlength{\parskip}{2ex}
\pagenumbering{gobble}

\usepackage[hang,flushmargin]{footmisc}
\usepackage[document]{ragged2e}

\begin{document}

Editors-in-Chief\hfill\today\\
\emph{Genetics}

Dear Editors:

We are pleased to submit our manuscript “Demes: a standard format for
demographic models” for your consideration for publication in \emph{Genetics}
as a Communications article.

The estimation and simulation of demographic models is central in population
genetics and genomics studies. Realistic demographic models are crucial for
downstream inferences of selection and genome biology, while understanding
species histories is itself a primary goal of the field. The amount of genomic
data and sophistication of inference and simulation tools now allow for the
estimation of increasingly complex demographic models, often involving dozens
of parameters with many interacting components.

However, this growing model complexity has revealed major hurdles to community
goals of replicability, robustness, and communication of
results, e.g.\footnote{Ragsdale, Aaron P., et al. ``Lessons learned from bugs in
models of human history.'' \emph{The American Journal of Human Genetics}
107.4 (2020): 583-588.}. These shortcomings have two primary causes: (1)
simulation and inference tools each have their own format for describing
demographic models, so that errors are common when translating input between
software, and (2) there are no agreed-upon guidelines for reporting demographic
models, so that published results are typically difficult to reconstruct and
are often presented with missing information.

Here, we present a solution to each of these issues. Demes defines a precise
and user-friendly format for demographic model specification, and we include a
suite of tools in multiple widely used programming languages for defining,
manipulating, and validating demographic models. Already, Demes has been
incorporated into popular simulation and inference tools, with growing support
among population genetics simulation developers.

From a user’s perspective, Demes-supported tools greatly reduce programming
burden and the possibility for errors and provide a concise and readable format
for sharing results. In addition to describing Demes, we showcase the
simplicity of specifying a multi-population model, which is used to illustrate
the model, simulate sampled genomes using \emph{msprime}, and compare to
expected patterns of diversity using \emph{moments}. Crucially, this requires
only a single specification of the demographic model, avoiding the need to
implement the model multiple time in different input formats.

We foresee Demes gaining wide use in the field, as it improves replicability of
published demographic inference results, allows interchange between a growing
number of simulation software and programming languages, and reduces potential
for user error and frustration of complex model implementation. Demes provides
a link between multiple popular population genetics simulation and inference
tools, including many that were published in \emph{Genetics} that maintain a
healthy community of users and developers. This project and paper should
therefore be of interest to readers of \emph{Genetics}.

Sincerely,\\
XXX, on behalf of all authors

\end{document}

\documentclass{letter}

\signature{Jerome Kelleher}

\address{Big Data Institute\\University of Oxford\\UK}
\begin{document}

\begin{letter}{GENETICS}

\opening{Dear Editors,}

I am writing on behalf of my coauthors to submit our manuscript entitled
\emph{Demes: a standard format for demographic models}
for consideration for publication in GENETICS
as a ``Methods, Technology, and Resources'' Communications article.

The estimation and simulation of demographic models is central to population
genetics and genomics studies.
With ever-growing datasets and recent improvements in simulation efficiency,
demographic models of interest can now involve dozens of parameters.
This growing model complexity has revealed major hurdles to community
goals of replicability, robustness, and communication of
results\footnote{Ragsdale, Aaron P., et al. ``Lessons learned from bugs in
models of human history.'' \emph{The American Journal of Human Genetics}
107.4 (2020): 583-588.}.
These shortcomings have two primary causes: (1)
simulation and inference tools each have their own format for describing
demographic models, so that errors are common when translating input between
software; and (2) there are no agreed-upon guidelines for reporting demographic
models, so that published results are difficult to reconstruct and
often presented with missing information.

Here we present a solution to these issues. Demes defines a precise
and user-friendly format for demographic model specification, and we include a
suite of tools in multiple widely used programming languages for defining,
manipulating, and validating demographic models. Already, Demes has been
incorporated into popular simulation and inference tools, with growing support
among population genetics tool developers.

We foresee Demes gaining wide use in the field, as it improves replicability of
published demographic inference results, allows interchange between
simulation and inference packages, and reduces potential
for user error. Demes provides
a link between multiple popular population genetics simulation and inference
tools, including many that were published in GENETICS that maintain a
healthy community of users and contibutors. We believe that this manuscript
will therefore
be of substantial interest to readers of GENETICS.

\closing{Sincerely,}

\end{letter}
\end{document}
